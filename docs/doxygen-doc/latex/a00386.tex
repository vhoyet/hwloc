 \hypertarget{a00386_attributes_normal}{}\section{Normal attributes}\label{a00386_attributes_normal}
hwloc objects have many generic attributes in the \hyperlink{a00238}{hwloc\+\_\+obj} structure, for instance their {\ttfamily logical\+\_\+index} or {\ttfamily os\+\_\+index} (see \hyperlink{a00394_faq_indexes}{Should I use logical or physical/\+OS indexes? and how?}), {\ttfamily depth} or {\ttfamily name}.

The kind of object is first described by the {\ttfamily obj-\/$>$type} generic attribute (an integer). OS devices also have a specific {\ttfamily obj-\/$>$attr-\/$>$osdev.\+type} integer for distinguishing between N\+I\+Cs, G\+P\+Us, etc. Objects may also have an optional {\ttfamily obj-\/$>$subtype} pointing to a better description string. For instance subtype is useful to say what Group objects are actually made of (e.\+g. {\itshape Book} for Linux S/390 books). It may also specify that a Block OS device is a {\itshape Disk}, or that a Co\+Processor OS device is a {\itshape C\+U\+DA} device. This subtype is displayed by lstopo either in place or after the main {\ttfamily obj-\/$>$type} attribute. N\+U\+MA nodes that correspond G\+PU memory may also have {\itshape G\+P\+U\+Memory} as subtype.

Each object also contains an {\ttfamily attr} field that, if non N\+U\+LL, points to a union \hyperlink{a00242}{hwloc\+\_\+obj\+\_\+attr\+\_\+u} of type-\/specific attribute structures. For instance, a L2\+Cache object {\ttfamily obj} contains cache-\/specific information in {\ttfamily obj-\/$>$attr-\/$>$cache}, such as its size and associativity, cache type. See \hyperlink{a00242}{hwloc\+\_\+obj\+\_\+attr\+\_\+u} for details.

 \hypertarget{a00386_attributes_info}{}\section{Custom string infos}\label{a00386_attributes_info}
Aside os these generic attribute fields, hwloc annotates many objects with string attributes that are made of a key and a value. Each object contains a list of such pairs that may be consulted manually (looking at the object {\ttfamily infos} array field) or using the \hyperlink{a00189_gab358661a92bb27d8542b255cc9f6f25e}{hwloc\+\_\+obj\+\_\+get\+\_\+info\+\_\+by\+\_\+name()}. The user may additionally add new key-\/value pairs to any object using \hyperlink{a00189_gace7654bb8a9002caae1a4b8a59e7452e}{hwloc\+\_\+obj\+\_\+add\+\_\+info()} or the \hyperlink{a00381_cli_hwloc_annotate}{hwloc-\/annotate} program.

Here is a non-\/exhaustive list of attributes that may be automatically added by hwloc. Note that these attributes heavily depend on the ability of the operating system to report them. Many of them will therefore be missing on some OS.

 \hypertarget{a00386_attributes_info_platform}{}\subsection{Hardware Platform Information}\label{a00386_attributes_info_platform}
These info attributes are attached to the root object (Machine).


\begin{DoxyDescription}
\item[Platform\+Name, Platform\+Model, Platform\+Vendor, Platform\+Board\+ID, Platform\+Revision, ]
\item[System\+Version\+Register, Processor\+Version\+Register (Machine) ]Some P\+O\+W\+E\+R/\+Power\+P\+C-\/specific attributes describing the platform and processor. Currently only available on Linux. Usually added to Package objects, but can be in Machine instead if hwloc failed to discover any package.  
\item[D\+M\+I\+Board\+Vendor, D\+M\+I\+Board\+Name, etc. ]D\+MI hardware information such as the motherboard and chassis models and vendors, the B\+I\+OS revision, etc., as reported by Linux under {\ttfamily /sys/class/dmi/id/}.  
\item[Memory\+Mode, Cluster\+Mode ]Intel Xeon Phi processor configuration modes. Available if hwloc-\/dump-\/hwdata was used (see \hyperlink{a00394_faq_knl_dump}{Why do I need hwloc-\/dump-\/hwdata for memory on Intel Xeon Phi processor?}) or if hwloc managed to guess them from the N\+U\+MA configuration.

The memory mode may be {\itshape Cache}, {\itshape Flat}, {\itshape Hybrid50} (half the M\+C\+D\+R\+AM is used as a cache) or {\itshape Hybrid25} (25\% of M\+C\+D\+R\+AM as cache). The cluster mode may be {\itshape Quadrant}, {\itshape Hemisphere}, {\itshape All2\+All}, {\itshape S\+N\+C2} or {\itshape S\+N\+C4}. See doc/examples/get-\/knl-\/modes.\+c in the source directory for an example of retrieving these attributes.  
\end{DoxyDescription}

 \hypertarget{a00386_attributes_info_os}{}\subsection{Operating System Information}\label{a00386_attributes_info_os}
These info attributes are attached to the root object (Machine).


\begin{DoxyDescription}
\item[O\+S\+Name, O\+S\+Release, O\+S\+Version, Host\+Name, Architecture ]The operating system name, release, version, the hostname and the architecture name, as reported by the Unix {\ttfamily uname} command.  
\item[Linux\+Cgroup ]The name the Linux control group where the calling process is placed.  
\item[Windows\+Build\+Environment ]Either Min\+GW or Cygwin when one of these environments was used during build.  
\end{DoxyDescription}

 \hypertarget{a00386_attributes_info_hwloc}{}\subsection{hwloc Information}\label{a00386_attributes_info_hwloc}
Unless specified, these info attributes are attached to the root object (Machine).


\begin{DoxyDescription}
\item[Backend (topology root, or specific object added by that backend) ]The name of the hwloc backend/component that filled the topology. If several components were combined, multiple Backend keys may exist, with different values, for instance {\ttfamily x86} and {\ttfamily Linux} in the root object and {\ttfamily C\+U\+DA} in C\+U\+DA OS device objects.  
\item[Synthetic\+Description ]The description string that was given to hwloc to build this synthetic topology.  
\item[hwloc\+Version ]The version number of the hwloc library that was used to generate the topology. If the topology was loaded from X\+ML, this is not the hwloc version that loaded it, but rather the first hwloc instance that exported the topology to X\+ML earlier.  
\item[Process\+Name ]The name of the process that contains the hwloc library that was used to generate the topology. If the topology was from X\+ML, this is not the hwloc process that loaded it, but rather the first process that exported the topology to X\+ML earlier.  
\end{DoxyDescription}

 \hypertarget{a00386_attributes_info_cpu}{}\subsection{C\+P\+U Information}\label{a00386_attributes_info_cpu}
These info attributes are attached to Package objects, or to the root object (Machine) if package locality information is missing.


\begin{DoxyDescription}
\item[C\+P\+U\+Model ]The processor model name. 
\item[C\+P\+U\+Vendor, C\+P\+U\+Model\+Number, C\+P\+U\+Family\+Number, C\+P\+U\+Stepping ]The processor vendor name, model number, family number, and stepping number. Currently available for x86 and Xeon Phi processors on most systems, and for ia64 processors on Linux (except C\+P\+U\+Stepping).  
\item[C\+P\+U\+Revision ]A P\+O\+W\+E\+R/\+Power\+P\+C-\/specific general processor revision number, currently only available on Linux.  
\item[C\+P\+U\+Type ]A Solaris-\/specific general processor type name, such as \char`\"{}i86pc\char`\"{}.  
\end{DoxyDescription}

 \hypertarget{a00386_attributes_info_osdev}{}\subsection{O\+S Device Information}\label{a00386_attributes_info_osdev}
These info attributes are attached to OS device objects specified in parentheses.


\begin{DoxyDescription}
\item[Vendor, Model, Revision, Serial\+Number, Size, Sector\+Size (Block OS devices) ]The vendor and model names, revision, serial number, size (in kB) and Sector\+Size (in bytes).  
\item[Linux\+Device\+ID (Block OS devices) ]The major/minor device number such as 8\+:0 of Linux device.  
\item[G\+P\+U\+Vendor, G\+P\+U\+Model (G\+PU or Co-\/\+Processor OS devices) ]The vendor and model names of the G\+PU device.  
\item[Open\+C\+L\+Device\+Type, Open\+C\+L\+Platform\+Index, ]
\item[Open\+C\+L\+Platform\+Name, Open\+C\+L\+Platform\+Device\+Index (Open\+CL OS devices) ]The type of Open\+CL device, the Open\+CL platform index and name, and the index of the device within the platform.  
\item[Open\+C\+L\+Compute\+Units, Open\+C\+L\+Global\+Memory\+Size (Open\+CL OS devices) ]The number of compute units and global memory size (in kB) of an Open\+CL device.  
\item[A\+M\+D\+U\+U\+ID, A\+M\+D\+Serial (R\+S\+MI G\+PU OS devices) ]The U\+U\+ID and serial number of A\+MD G\+P\+Us.  
\item[X\+G\+M\+I\+Hive\+ID (R\+S\+MI G\+PU OS devices) ]The ID of the group of G\+P\+Us (Hive) interconnected by X\+G\+MI links  
\item[X\+G\+M\+I\+Peers (R\+S\+MI G\+PU OS devices) ]The list of R\+S\+MI OS devices that are directly connected to the current device through X\+G\+MI links. They are given as a space-\/separated list of object names, for instance {\itshape rsmi2 rsmi3}.  
\item[N\+V\+I\+D\+I\+A\+U\+U\+ID, N\+V\+I\+D\+I\+A\+Serial (N\+V\+ML G\+PU OS devices) ]The U\+U\+ID and serial number of N\+V\+I\+D\+IA G\+P\+Us.  
\item[C\+U\+D\+A\+Multi\+Processors, C\+U\+D\+A\+Cores\+Per\+MP, ]
\item[C\+U\+D\+A\+Global\+Memory\+Size, C\+U\+D\+A\+L2\+Cache\+Size, C\+U\+D\+A\+Shared\+Memory\+Size\+Per\+MP (C\+U\+DA OS devices) ]The number of shared multiprocessors, the number of cores per multiprocessor, the global memory size, the (global) L2 cache size, and size of the shared memory in each multiprocessor of a C\+U\+DA device. Sizes are in kB.  
\item[Address, Port (Network interface OS devices) ]The M\+AC address and the port number of a software network interface, such as {\ttfamily eth4} on Linux.  
\item[Node\+G\+U\+ID, Sys\+Image\+G\+U\+ID, Port1\+State, Port2\+L\+ID, Port2\+L\+MC, Port3\+G\+I\+D1 (Open\+Fabrics OS devices) ]The node G\+U\+ID and G\+U\+ID mask, the state of a port \#1 (value is 4 when active), the L\+ID and L\+ID mask count of port \#2, and G\+ID \#1 of port \#3.  
\end{DoxyDescription}

 \hypertarget{a00386_attributes_info_otherobjs}{}\subsection{Other Object-\/specific Information}\label{a00386_attributes_info_otherobjs}
These info attributes are attached to objects specified in parentheses.


\begin{DoxyDescription}
\item[D\+A\+X\+Device (N\+U\+MA Nodes) ]The name of the Linux D\+AX device that was used to expose a non-\/volatile memory region as a volatile N\+U\+MA node.  
\item[P\+C\+I\+Bus\+ID (G\+P\+U\+Memory N\+U\+MA Nodes) ]The P\+CI bus ID of the G\+PU whose memory is exposed in this N\+U\+MA node.  
\item[Inclusive (Caches) ]The inclusiveness of a cache (1 if inclusive, 0 otherwise). Currently only available on x86 processors.  
\item[Solaris\+Processor\+Group (Group) ]The Solaris kstat processor group name that was used to build this Group object.  
\item[P\+C\+I\+Vendor, P\+C\+I\+Device (P\+CI devices and bridges) ]The vendor and device names of the P\+CI device.  
\item[P\+C\+I\+Slot (P\+CI devices or Bridges) ]The name/number of the physical slot where the device is plugged. If the physical device contains P\+CI bridges above the actual P\+CI device, the attribute may be attached to the highest bridge (i.\+e. the first object that actually appears below the physical slot).  
\item[Vendor, Asset\+Tag, Part\+Number, Device\+Location, Bank\+Location (Memory\+Module Misc objects) ]Information about memory modules (D\+I\+M\+Ms) extracted from S\+M\+B\+I\+OS.  
\end{DoxyDescription}

 \hypertarget{a00386_attributes_info_user}{}\subsection{User-\/\+Given Information}\label{a00386_attributes_info_user}
Here is a non-\/exhaustive list of user-\/provided info attributes that have a special meaning\+: 
\begin{DoxyDescription}
\item[lstopo\+Style ]Enforces the style of an object (background and text colors) in the graphical output of lstopo. See C\+U\+S\+T\+OM C\+O\+L\+O\+RS in the lstopo(1) manpage for details.  
\end{DoxyDescription}